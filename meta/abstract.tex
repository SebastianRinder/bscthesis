\begin{abstract}[1]
    Reinforcement learning relies on policy gradient but the gradient is known only in expectation and most of the time stochastic policies. This leaves some room for zero order methods and BO can combine solving the problem and the exploration strategy from deterministic policies.
    We investigate in this paper how to integrate efficient exploration strategies stemming from Bayesian optimization for solving high dimensional reinforcement learning problems. We propose a novel optimization algorithm that is able to scale Bayesian optimization to such high dimensional tasks by restricting the search to the local vicinity of a search distribution and by proposing kernels capturing similarity in behavior rather than parameter. We show in the experiments that our approach can be very useful for applications such as robotics.
\end{abstract}
%
%
\selectlanguage{ngerman} % select german language
\begin{abstract}[2]
    Das Ziel im bestärkten Lernen ist das Finden einer Strategie, welche die erhaltene Belohnung eines Agenten maximiert. Da der Suchraum für mögliche Strategien sehr groß sein kann, verwenden wir Bayesian optimization, um die Anzahl der Evaluierungen durch den Agenten zu minimieren. Das hat den Vorteil, dass zeit- und kostenaufwändige Abläufe, wie beispielsweise das Bewegen eines Roboterarms, reduziert werden.
    Die Effektivität der Suche wird maßgeblich von der Wahl des Kernels beeinflusst.
    Standardkernel in der Bayesian optimization vergleichen die Parameter von Strategien um eine Vorhersage über bisher nicht evaluierte Strategien zu treffen.

    Der Trajectorykernel vergleicht statt der Parameter, die aus den jeweiligen Strategien resultierenden Verhaltensweisen. Dadurch werden unterschiedliche Strategien mit ähnlichem Resultat von der Suche weniger priorisiert.

    Wir zeigen die Überlegenheit des verhaltensbasierten Kernels gegenüber dem parameterbasierten anhand von Robotersteruerungssimulationen.
\end{abstract}
\selectlanguage{english} % reset to english language
