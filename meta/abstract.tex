\begin{abstract}[1]
    Bayesian optimization for solving reinforcement learning problems has a few drawbacks. Firstly, it only detects similarities in policy parameters, not in behaviors generated by the reinforcement learning agent. And secondly, it does not scale well to higher dimensional problems due to the huge search space that needs to be optimized globally. In this thesis we investigate how trajectory kernels, kernels that exploit trajectory data generated by the reinforcement learning agent, apply to Bayesian optimization in a robotic task. And we show that restricting the Bayesian optimization search area to a local vicinity enhances the learning performance in such robotic reinforcement learning tasks. For the local Bayesian optimization, adapted from \cite{akrour2017local}, we also need a suitable trajectory kernel, which we contribute. The trajectory kernel for global Bayesian optimization is adapted from \cite{wilson2014using}. We show local Blah ist geiler als global Blah.
\end{abstract}




% \selectlanguage{ngerman} % select german language
% \begin{abstract}[2]
%     Das Ziel im bestärkten Lernen ist das Finden einer Strategie, welche die erhaltene Belohnung eines Agenten maximiert. Da der Suchraum für mögliche Strategien sehr groß sein kann, verwenden wir Bayesian optimization, um die Anzahl der Evaluierungen durch den Agenten zu minimieren. Das hat den Vorteil, dass zeit- und kostenaufwändige Abläufe, wie beispielsweise das Bewegen eines Roboterarms, reduziert werden.
%     Die Effektivität der Suche wird maßgeblich von der Wahl des Kernels beeinflusst.
%     Standardkernel in der Bayesian optimization vergleichen die Parameter von Strategien um eine Vorhersage über bisher nicht evaluierte Strategien zu treffen.
%
%     Der Trajectorykernel vergleicht statt der Parameter, die aus den jeweiligen Strategien resultierenden Verhaltensweisen. Dadurch werden unterschiedliche Strategien mit ähnlichem Resultat von der Suche weniger priorisiert.
%
%     Wir zeigen die Überlegenheit des verhaltensbasierten Kernels gegenüber dem parameterbasierten anhand von Robotersteruerungssimulationen.
% \end{abstract}
% \selectlanguage{english} % reset to english language
